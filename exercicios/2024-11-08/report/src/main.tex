\documentclass{article}
\usepackage[utf8]{inputenc}
\usepackage[margin=1in]{geometry}
\usepackage[all]{xy}
\usepackage[brazil]{babel}   
\usepackage[section]{placeins}

\usepackage{amsmath,amsthm,amssymb,color,latexsym}
\usepackage{geometry}        
\geometry{letterpaper}    
\usepackage{graphicx}
\usepackage{minted}

\begin{document}

\noindent DFS \hfill ELC120 - Projeto e Análise de Algoritmos \\
Jaime Antonio Daniel Filho (14/11/2024)

\noindent \hrulefill
\begin{itemize}
    \item Implementar e testar o algoritmo DFS. \\
    
    Em anexo.
    
    \item Os dois pseudocódigos acima (recursivo e com pilha explícita), processam as listas de adjacência de um nodo na mesma ordem? Justifique. \\

    O pseudocódigo com pilha explícita explora as arestas de um nodo na ordem inversa da lista de adjacência, enquanto o pseudocódigo recursivo as explora na mesma ordem em que aparecem.
    
    Isso ocorre porque, no pseudocódigo com pilha explícita, as arestas do nodo são inseridas na pilha em sequência, mas, como a pilha opera no modo LIFO (Last In, First Out), a última aresta inserida será a primeira a ser explorada. Isso significa que as arestas são processadas na ordem inversa àquela em que foram inseridas.
    
    Por outro lado, no pseudocódigo recursivo, as arestas são exploradas imediatamente na ordem em que aparecem na lista de adjacência. Cada chamada recursiva segue diretamente para a próxima aresta, mantendo a ordem original de exploração.
    
    \item Qual a complexidade de tempo do procedimento DFS? Justifique. \\

    A complexidade de tempo do DFS é $O(V + E)$, pois ele visita cada vértice $V$ exatamente uma única vez e percorre cada aresta $E$ no máximo duas vezes (uma para cada extremidade), assumindo uma representação do grafo por listas de adjacências.
    
\end{itemize}

\end{document}